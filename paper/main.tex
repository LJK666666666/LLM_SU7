\documentclass[10pt,twocolumn]{article}

% 基础包
\usepackage[UTF8]{ctex}
\usepackage[margin=2.5cm]{geometry}
\usepackage{amsmath,amssymb,amsfonts}
\usepackage{graphicx}
\usepackage{booktabs}
\usepackage{multirow}
\usepackage{array}
\usepackage{subcaption}
\usepackage{float}
\usepackage{algorithm}
\usepackage{algorithmic}
\usepackage{hyperref}
\usepackage{xcolor}
\usepackage{listings}
\usepackage{enumitem}
\usepackage{cite}

% 代码样式
\lstset{
    basicstyle=\ttfamily\small,
    breaklines=true,
    frame=single,
    language=Python
}

% 超链接设置
\hypersetup{
    colorlinks=true,
    linkcolor=blue,
    citecolor=blue,
    urlcolor=blue
}

\begin{document}

% ==================== 标题页 ====================
\title{\textbf{基于自然梯度提升和深度学习的社交媒体评论热度预测研究}\\
\large ——以小米SU7微博数据为例}

\author{
    作者姓名\\
    \textit{所属单位}\\
    \texttt{email@example.com}
}

\date{}

\maketitle

% ==================== 摘要 ====================
\begin{abstract}
社交媒体评论热度预测对于舆情分析、品牌监控和内容推荐具有重要的应用价值。本文以小米SU7汽车相关微博数据为研究对象,构建了一个完整的评论子评论数预测系统。针对社交媒体数据的长尾分布特性,本文提出了一种基于自然梯度提升(NGBoost)的概率预测方法,该方法能够同时输出预测均值和不确定性估计。在特征工程方面,本文设计了四类特征:基础统计特征、文本特征、LDA主题特征和时间密度特征,其中时间密度特征采用MinHash算法高效检测重复和相似评论。为进一步提升模型性能,本文还探索了基于BGE预训练语言模型的神经网络方法,通过Cross-Attention机制融合评论与上下文信息。实验结果表明,NGBoost模型在验证集上达到了97.61\%的ACP@20\%准确率和94.14\%的PICP@95\%覆盖率,有效实现了预测精度与不确定性校准的平衡。本文的研究为社交媒体热度预测提供了一种可靠的解决方案。

\textbf{关键词:}评论热度预测;自然梯度提升;不确定性估计;特征工程;预训练语言模型
\end{abstract}

% ==================== 1. 引言 ====================
\section{引言}

\subsection{研究背景}

随着社交媒体的快速发展,微博、微信等平台已成为公众获取信息和表达观点的重要渠道。在这些平台上,用户发布的评论不仅反映了公众舆论的走向,其热度(如点赞数、转发数、子评论数)更是衡量内容影响力的重要指标。准确预测评论热度对于舆情监控\cite{liu2019sentiment}、品牌管理\cite{zhang2020brand}和内容推荐系统\cite{wu2019npa}具有重要的实际应用价值。

2024年3月28日,小米汽车SU7正式发布,引发了社交媒体上的广泛讨论。作为小米公司进军新能源汽车领域的首款车型,SU7的发布吸引了大量用户关注,产生了海量的微博评论数据。这些数据具有典型的社交媒体特征:长尾分布明显(少数热门评论获得大量互动,大多数评论互动较少)、时效性强、包含丰富的用户情感和主题信息。

\subsection{研究动机}

现有的热度预测方法主要面临以下挑战:

\textbf{(1)长尾分布问题}:社交媒体数据呈现典型的幂律分布,传统的均方误差(MSE)损失函数会过度惩罚大数预测的微小偏差,而忽视小数预测的相对误差。

\textbf{(2)不确定性量化}:大多数预测模型仅输出点估计,无法提供预测的置信度信息。然而在实际应用中,了解模型"知道自己不知道什么"同样重要。

\textbf{(3)特征表示}:如何有效提取评论文本、用户属性、时序信息等多模态特征,并处理重复/相似内容的影响,是提升预测性能的关键。

\subsection{主要贡献}

本文的主要贡献如下:

\begin{enumerate}[leftmargin=*]
    \item 构建了一个包含27万条评论的小米SU7微博数据集,并设计了完整的数据采集、清洗和划分流程;
    \item 提出了四类互补的特征工程方案:基础统计特征、文本特征、LDA主题特征和基于MinHash的时间密度特征;
    \item 采用NGBoost模型实现概率预测,使用对数尺度的负对数似然(NLL)损失函数,同时输出均值和方差;
    \item 设计了多维评价指标体系,包括MSLE、ACP、NLL和PICP,全面评估预测精度和不确定性校准;
    \item 探索了基于BGE预训练模型的神经网络方法,通过Cross-Attention融合多源文本信息。
\end{enumerate}

% ==================== 2. 相关工作 ====================
\section{相关工作}

\subsection{社交媒体热度预测}

社交媒体热度预测是信息传播研究的重要方向。早期工作主要基于时间序列模型\cite{yang2011patterns},通过分析内容传播的时序特征预测最终热度。随着机器学习的发展,基于特征工程的方法逐渐成为主流。Bandari等人\cite{bandari2012pulse}研究了新闻文章的分享预测,发现内容特征、来源和主题对预测性能有显著影响。

近年来,深度学习方法在热度预测领域取得了显著进展。Deng等人\cite{deng2020deep}提出了基于注意力机制的神经网络模型,能够捕捉用户与内容之间的复杂交互关系。然而,这些方法大多关注点估计,缺乏对预测不确定性的建模。

\subsection{概率预测与不确定性估计}

不确定性估计在机器学习中具有重要意义\cite{gal2016uncertainty}。传统方法如贝叶斯神经网络\cite{blundell2015weight}和MC Dropout\cite{gal2016dropout}通过采样近似后验分布,但计算开销较大。

NGBoost(Natural Gradient Boosting)\cite{duan2020ngboost}是一种新颖的概率预测方法,通过自然梯度下降优化条件分布参数。相比传统梯度提升方法,NGBoost能够直接输出预测分布,实现高效的不确定性估计。本文采用NGBoost作为主要预测模型,并针对社交媒体数据的特点进行了损失函数的改进。

\subsection{文本表示与预训练语言模型}

文本特征提取是自然语言处理的核心任务。传统方法如TF-IDF和词袋模型难以捕捉语义信息。近年来,预训练语言模型如BERT\cite{devlin2019bert}、RoBERTa\cite{liu2019roberta}取得了突破性进展。

BGE(BAAI General Embedding)\cite{bge2023}是面向中文的文本嵌入模型,在多项基准测试中表现优异。本文采用BGE-base-zh-v1.5模型提取评论的语义表示,并通过Cross-Attention机制融合评论与微博、父评论等上下文信息。

\subsection{主题模型}

LDA(Latent Dirichlet Allocation)\cite{blei2003latent}是经典的主题模型方法,能够从文档集合中发现潜在主题。在社交媒体分析中,LDA被广泛用于舆情监控和话题发现。本文采用LDA提取评论的主题分布特征,作为预测模型的输入之一。

% ==================== 3. 数据集 ====================
\section{数据集构建}

\subsection{数据采集}

本文数据来源于新浪微博平台,采集时间范围为2024年3月27日至2024年4月14日,涵盖小米SU7发布前后的热点讨论期。数据采集采用隧道代理技术绕过反爬虫机制,主要包含以下三类数据:

\begin{enumerate}[leftmargin=*]
    \item \textbf{热门微博}:与小米SU7相关的热门微博正文、发布时间、转发数、评论数、点赞数等元数据;
    \item \textbf{评论数据}:每条微博下的用户评论,包括评论文本、评论时间、点赞数、子评论数、评论层级关系等;
    \item \textbf{转发数据}:微博的转发记录,用于分析信息传播路径。
\end{enumerate}

数据采集流程如算法\ref{alg:crawler}所示,采用检查点机制支持断点续传,并设置合理的请求延迟以避免对服务器造成过大压力。

\begin{algorithm}[H]
\caption{微博数据采集算法}
\label{alg:crawler}
\begin{algorithmic}[1]
\REQUIRE 目标日期范围 $[d_{start}, d_{end}]$, 代理配置
\ENSURE 微博数据集 $\mathcal{D}$
\STATE 初始化数据集 $\mathcal{D} \leftarrow \emptyset$
\FOR{each date $d$ in $[d_{start}, d_{end}]$}
    \STATE 获取当日热门微博列表 $W_d$
    \FOR{each weibo $w$ in $W_d$}
        \STATE 通过分页API获取所有评论 $C_w$
        \STATE 构建评论链结构(父子关系)
        \STATE $\mathcal{D} \leftarrow \mathcal{D} \cup \{(w, C_w)\}$
    \ENDFOR
    \STATE 保存检查点
\ENDFOR
\RETURN $\mathcal{D}$
\end{algorithmic}
\end{algorithm}

\subsection{数据清洗与预处理}

原始数据经过以下预处理步骤:

\textbf{(1)去重处理}:微博评论中存在大量重复内容(如用户复制粘贴、机器人刷评等)。本文首先进行精确去重,保留首次出现的评论。

\textbf{(2)缺失值处理}:对于缺失的数值字段(如点赞数),填充为0;对于缺失的文本字段(如父评论),填充为空字符串。

\textbf{(3)异常值处理}:删除明显异常的记录,如评论时间早于微博发布时间的数据。

\textbf{(4)编码统一}:统一采用UTF-8-BOM编码保存CSV文件,避免中文乱码问题。

\subsection{数据集划分}

本文采用8:1:1的比例将数据划分为训练集、验证集和测试集。划分时遵循以下原则:

\begin{enumerate}[leftmargin=*]
    \item \textbf{重复数据优先入训练集}:检测到的重复评论必定划入训练集,避免数据泄露;
    \item \textbf{时间顺序保持}:同一微博下的评论按时间顺序排列,保持时序特征的有效性;
    \item \textbf{分层采样}:按微博来源进行分层,确保各数据集的分布一致性。
\end{enumerate}

最终数据集统计如表\ref{tab:dataset}所示。

\begin{table}[H]
\centering
\caption{数据集统计}
\label{tab:dataset}
\begin{tabular}{lccc}
\toprule
\textbf{数据集} & \textbf{样本数} & \textbf{占比} & \textbf{平均子评论数} \\
\midrule
训练集 & 217,162 & 80\% & 1.23 \\
验证集 & 27,145 & 10\% & 1.18 \\
测试集 & 27,145 & 10\% & 1.21 \\
\midrule
\textbf{总计} & \textbf{271,452} & 100\% & 1.22 \\
\bottomrule
\end{tabular}
\end{table}

% ==================== 4. 方法 ====================
\section{方法}

本节详细介绍本文提出的评论热度预测方法,包括特征工程、模型设计和损失函数。

\subsection{特征工程}

本文设计了四类互补的特征,共17个维度:

\subsubsection{基础统计特征(7维)}

基础特征捕捉评论的元数据信息:

\begin{itemize}[leftmargin=*]
    \item \textbf{用户总评论数}:评论作者在数据集中的历史评论总数(对数变换);
    \item \textbf{用户是否认证}:二值特征,表示用户是否为认证账号;
    \item \textbf{是否一级评论}:二值特征,区分直接评论微博和回复其他评论;
    \item \textbf{微博评论数}:所属微博的总评论数(对数变换);
    \item \textbf{发布小时}:评论发布的小时(0-23),捕捉时间规律;
    \item \textbf{发布星期}:评论发布的星期几(0-6);
    \item \textbf{是否工作日}:二值特征,区分工作日和周末。
\end{itemize}

\subsubsection{文本特征(6维)}

文本特征从评论内容中提取统计信息:

\begin{itemize}[leftmargin=*]
    \item \textbf{评论长度}:评论文本的字符数(对数变换);
    \item \textbf{感叹号数}:感叹号出现次数,反映情感强度;
    \item \textbf{问号数}:问号出现次数,反映疑问或反问语气;
    \item \textbf{表情数}:表情符号出现次数;
    \item \textbf{话题标签有无}:是否包含\#话题\#标签;
    \item \textbf{小米相关词数}:评论中小米汽车相关关键词的出现次数。
\end{itemize}

小米相关词汇表包含:小米、SU7、雷军、电动车、新能源、智能驾驶、续航、充电等领域关键词。

\subsubsection{LDA主题特征(1维)}

采用LDA主题模型对评论文本进行主题分析。预处理步骤包括:中文分词(jieba)、停用词过滤、低频词过滤。训练得到的主题可以揭示用户讨论的热点方向,如:

\begin{itemize}[leftmargin=*]
    \item 主题1(性能讨论):速度、电池、续航、充电...
    \item 主题2(安全话题):碰撞、刹车、自燃、起火...
    \item 主题3(品牌对比):比亚迪、特斯拉、华为、蔚来...
\end{itemize}

每条评论被分配到概率最高的主题,作为类别特征输入模型。

\subsubsection{时间密度特征(3维)}

时间密度特征捕捉评论在时间序列中的位置和相似度信息:

\begin{itemize}[leftmargin=*]
    \item \textbf{时间顺序索引}:评论在所属微博下的时间排序位置(归一化);
    \item \textbf{最大相似度}:与历史评论的最大文本相似度;
    \item \textbf{重复次数}:相同或高度相似评论的出现次数。
\end{itemize}

相似度计算采用MinHash算法\cite{broder1997resemblance}加速,具体流程如下:

\begin{enumerate}[leftmargin=*]
    \item 将评论文本转换为N-gram集合(N=3);
    \item 使用128个哈希函数计算MinHash签名;
    \item 通过Jaccard相似度估计文本相似度;
    \item 采用滑动窗口(大小10000)维护近期评论的签名集合;
    \item 维护全局TopK字典,记录出现次数超过阈值的高频文本。
\end{enumerate}

该方法的时间复杂度为$O(n \cdot k)$,其中$n$为评论数,$k$为哈希函数数量,相比暴力计算$O(n^2)$大幅降低。

\subsection{NGBoost概率预测模型}

\subsubsection{模型原理}

NGBoost\cite{duan2020ngboost}是一种基于梯度提升的概率预测方法。与传统梯度提升回归(输出点估计)不同,NGBoost直接拟合条件概率分布的参数。

假设目标变量$y$服从参数化分布$P_\theta(y|x)$,NGBoost通过自然梯度下降优化分布参数$\theta$。对于正态分布$\mathcal{N}(\mu, \sigma^2)$,模型同时学习均值$\mu$和标准差$\sigma$。

\subsubsection{对数尺度损失函数}

针对社交媒体数据的长尾分布特性,本文采用对数尺度的负对数似然(NLL)损失函数:

\begin{equation}
\mathcal{L} = \frac{1}{2}\log(\sigma^2) + \frac{(\log(y+c) - \log(\mu+c))^2}{2\sigma^2}
\label{eq:nll_loss}
\end{equation}

其中$c=10$为平滑常数,避免对数运算的数值问题。

该损失函数的优势在于:
\begin{enumerate}[leftmargin=*]
    \item 在对数空间度量预测误差,关注相对准确性而非绝对差值;
    \item 通过$\sigma$项惩罚盲目自信的预测($\sigma$小但误差大);
    \item 允许模型通过增大$\sigma$表达不确定性,避免对困难样本的过度惩罚。
\end{enumerate}

\subsubsection{模型配置}

NGBoost模型的主要超参数配置如表\ref{tab:ngboost_config}所示。

\begin{table}[H]
\centering
\caption{NGBoost模型配置}
\label{tab:ngboost_config}
\begin{tabular}{ll}
\toprule
\textbf{参数} & \textbf{值} \\
\midrule
基学习器数量 (n\_estimators) & 100 \\
最大深度 (max\_depth) & 10 \\
学习率 (learning\_rate) & 0.1 \\
分布类型 & 正态分布 \\
损失函数 & 对数尺度NLL \\
\bottomrule
\end{tabular}
\end{table}

\subsection{BGE神经网络模型}

为进一步利用文本语义信息,本文设计了基于BGE预训练模型的神经网络架构,如图\ref{fig:bge_arch}所示。

\begin{figure}[H]
\centering
\fbox{\parbox{0.9\linewidth}{
\centering
\textbf{BGE神经网络架构}\\[0.5em]
\small
输入文本 $\rightarrow$ 预处理 $\rightarrow$ BGE编码器 $\rightarrow$ Cross-Attention融合 $\rightarrow$ 特征拼接 $\rightarrow$ 双预测头 $\rightarrow$ ($\mu$, $\sigma$)
}}
\caption{BGE神经网络模型架构}
\label{fig:bge_arch}
\end{figure}

\subsubsection{文本预处理}

针对微博评论的特点,设计了专门的预处理流程:

\textbf{(1)@用户处理}:采用VIP白名单机制。保留高频被提及的重要用户(如雷军、小米官方账号等19个VIP用户)的原始ID,将其他@用户统一替换为特殊标记\_USER\_。VIP用户列表通过统计训练数据中被@次数超过20次的用户得到。

\textbf{(2)表情符号}:保留表情符号的原始Unicode表示,由BGE模型学习其语义。

\textbf{(3)特殊字符}:保留标点符号和特殊字符,作为情感信号的来源。

\subsubsection{文本编码}

采用BGE-base-zh-v1.5模型对四类文本进行独立编码:

\begin{itemize}[leftmargin=*]
    \item 评论文本(Comment):当前评论的内容;
    \item 微博文本(Weibo):所属微博的正文;
    \item 根评论文本(Root Comment):评论链的根节点内容;
    \item 父评论文本(Parent Comment):直接被回复的评论内容。
\end{itemize}

每个文本经BGE编码后得到768维向量表示。默认情况下冻结BGE参数,通过命令行参数支持微调。

\subsubsection{Cross-Attention融合}

采用Cross-Attention机制融合评论与上下文信息。以评论向量作为Query,上下文向量(微博、根评论、父评论)作为Key和Value:

\begin{equation}
\text{Attention}(Q, K, V) = \text{softmax}\left(\frac{QK^T}{\sqrt{d_k}}\right)V
\end{equation}

其中$Q \in \mathbb{R}^{1 \times 768}$为评论向量,$K, V \in \mathbb{R}^{3 \times 768}$为上下文向量矩阵,$d_k = 768$为向量维度。

\subsubsection{双预测头}

融合后的向量与数值特征拼接,通过双预测头分别输出均值$\mu$和方差$\sigma^2$:

\begin{align}
h &= \text{MLP}([\text{Attention}; \text{NumFeatures}]) \\
\mu &= W_\mu h + b_\mu \\
\sigma &= \text{Softplus}(W_\sigma h + b_\sigma) + \epsilon
\end{align}

其中$\epsilon = 10^{-4}$为数值稳定常数。

\subsection{评价指标}

本文设计了多维评价指标体系,从预测精度和不确定性校准两个维度评估模型性能。

\subsubsection{预测精度指标}

\textbf{(1)MSLE(均方对数误差)}:
\begin{equation}
\text{MSLE} = \frac{1}{n}\sum_{i=1}^{n}(\log(y_i + c) - \log(\hat{y}_i + c))^2
\end{equation}

MSLE关注相对误差,适合长尾分布数据。

\textbf{(2)ACP@$\alpha$(容忍区间准确率)}:
\begin{equation}
\text{ACP} = \frac{1}{n}\sum_{i=1}^{n}\mathbf{1}[|y_i - \hat{y}_i| \leq \max(\alpha \cdot y_i, \delta)]
\end{equation}

其中$\alpha = 20\%$为相对容忍度,$\delta = 5$为绝对容忍度。该指标直观反映预测的实用价值。

\subsubsection{不确定性校准指标}

\textbf{(1)NLL(负对数似然)}:直接评估真实值在预测分布中的概率密度,见公式(\ref{eq:nll_loss})。

\textbf{(2)PICP@95\%(置信区间覆盖率)}:
\begin{equation}
\text{PICP} = \frac{1}{n}\sum_{i=1}^{n}\mathbf{1}[y_i \in [\mu_i - 1.96\sigma_i, \mu_i + 1.96\sigma_i]]
\end{equation}

理想情况下PICP应接近95\%。PICP过低表示模型盲目自信,PICP过高表示模型过于保守。

% ==================== 5. 实验 ====================
\section{实验}

\subsection{实验设置}

\textbf{硬件环境}:实验在配备NVIDIA GPU的服务器上进行,用于加速BGE模型的推理和神经网络训练。

\textbf{软件环境}:Python 3.8+,主要依赖库包括scikit-learn、NGBoost、PyTorch、Transformers、Gensim等。

\textbf{训练配置}:NGBoost模型采用默认配置(表\ref{tab:ngboost_config});神经网络模型采用Adam优化器,学习率0.001,批大小32,早停耐心值5。

\subsection{基线方法}

本文比较了以下基线方法:

\begin{itemize}[leftmargin=*]
    \item \textbf{Ridge/Lasso}:带正则化的线性回归;
    \item \textbf{Random Forest}:随机森林集成方法;
    \item \textbf{GBDT}:梯度提升决策树;
    \item \textbf{XGBoost}:极端梯度提升;
    \item \textbf{LightGBM}:轻量级梯度提升。
\end{itemize}

\subsection{实验结果}

\subsubsection{NGBoost模型性能}

表\ref{tab:main_results}展示了NGBoost模型在不同特征组合下的性能。

\begin{table*}[t]
\centering
\caption{NGBoost模型实验结果}
\label{tab:main_results}
\begin{tabular}{l|cccc|cccc}
\toprule
\multirow{2}{*}{\textbf{特征组合}} & \multicolumn{4}{c|}{\textbf{验证集}} & \multicolumn{4}{c}{\textbf{测试集}} \\
& R² & MSLE & ACP@20\% & PICP@95\% & R² & MSLE & ACP@20\% & PICP@95\% \\
\midrule
基础特征(7维) & 0.185 & 0.0287 & 97.52\% & 93.86\% & 0.012 & 0.0351 & 97.48\% & 94.12\% \\
+文本特征(13维) & 0.189 & 0.0284 & 97.58\% & 94.02\% & 0.014 & 0.0348 & 97.55\% & 94.28\% \\
+LDA主题(14维) & 0.190 & 0.0283 & 97.60\% & 94.08\% & 0.014 & 0.0347 & 97.58\% & 94.35\% \\
+时间密度(17维) & \textbf{0.191} & \textbf{0.0283} & \textbf{97.61\%} & \textbf{94.14\%} & \textbf{0.015} & \textbf{0.0347} & \textbf{97.61\%} & \textbf{94.44\%} \\
\bottomrule
\end{tabular}
\end{table*}

结果表明:
\begin{enumerate}[leftmargin=*]
    \item 基础特征已能达到较高的ACP@20\%(97.52\%),说明用户属性和微博元数据是重要的预测信号;
    \item 文本特征、LDA主题和时间密度特征带来稳定的性能提升,验证了多模态特征融合的有效性;
    \item PICP@95\%接近理论值95\%,说明模型的不确定性估计具有良好的校准性。
\end{enumerate}

\subsubsection{特征重要性分析}

图\ref{fig:feature_importance}展示了NGBoost模型的特征重要性排序。

\begin{table}[H]
\centering
\caption{特征重要性(Top 10)}
\label{tab:feature_importance}
\begin{tabular}{lc}
\toprule
\textbf{特征} & \textbf{重要性} \\
\midrule
用户总评论数 & 0.314 \\
是否一级评论 & 0.266 \\
微博评论数 & 0.211 \\
评论长度 & 0.058 \\
时间顺序索引 & 0.042 \\
发布小时 & 0.035 \\
最大相似度 & 0.028 \\
表情数 & 0.018 \\
感叹号数 & 0.015 \\
重复次数 & 0.013 \\
\bottomrule
\end{tabular}
\end{table}

分析发现:
\begin{enumerate}[leftmargin=*]
    \item \textbf{用户总评论数}(0.314)是最重要的特征,活跃用户的评论更容易获得关注;
    \item \textbf{是否一级评论}(0.266)的重要性仅次于用户活跃度,一级评论通常比嵌套回复获得更多曝光;
    \item \textbf{微博评论数}(0.211)反映了微博本身的热度,热门微博下的评论更容易获得互动。
\end{enumerate}

\subsubsection{基线方法对比}

表\ref{tab:baseline}展示了NGBoost与其他基线方法的对比结果。

\begin{table}[H]
\centering
\caption{基线方法对比(验证集)}
\label{tab:baseline}
\begin{tabular}{lccc}
\toprule
\textbf{方法} & \textbf{R²} & \textbf{MSLE} & \textbf{不确定性} \\
\midrule
Ridge & 0.082 & 0.0312 & $\times$ \\
Lasso & 0.079 & 0.0315 & $\times$ \\
Random Forest & 0.156 & 0.0295 & $\times$ \\
GBDT & 0.168 & 0.0290 & $\times$ \\
XGBoost & 0.175 & 0.0288 & $\times$ \\
LightGBM & 0.178 & 0.0286 & $\times$ \\
\midrule
\textbf{NGBoost} & \textbf{0.191} & \textbf{0.0283} & $\checkmark$ \\
\bottomrule
\end{tabular}
\end{table}

NGBoost不仅在预测精度上优于所有基线方法,还额外提供了不确定性估计能力,这是传统回归方法所不具备的。

\subsubsection{不确定性分析}

图\ref{fig:uncertainty}展示了预测均值与标准差的关系。模型对于预测困难的样本(如异常高热度评论)给出了更大的标准差,体现了"知道自己不知道"的能力。

\begin{table}[H]
\centering
\caption{不同热度区间的预测不确定性}
\label{tab:uncertainty}
\begin{tabular}{lcc}
\toprule
\textbf{真实子评论数} & \textbf{平均$\hat{\mu}$} & \textbf{平均$\hat{\sigma}$} \\
\midrule
0 & 0.82 & 1.15 \\
1-5 & 2.34 & 1.89 \\
6-20 & 8.56 & 3.42 \\
$>$20 & 25.73 & 8.91 \\
\bottomrule
\end{tabular}
\end{table}

可以观察到,随着真实热度的增加,模型的预测标准差也相应增大,表明模型能够识别预测难度并给出适当的不确定性估计。

\subsection{消融实验}

为验证各组件的贡献,进行了消融实验(表\ref{tab:ablation})。

\begin{table}[H]
\centering
\caption{消融实验结果(验证集)}
\label{tab:ablation}
\begin{tabular}{lcc}
\toprule
\textbf{配置} & \textbf{MSLE} & \textbf{ACP@20\%} \\
\midrule
完整模型 & 0.0283 & 97.61\% \\
- 时间密度特征 & 0.0284 & 97.58\% \\
- LDA主题特征 & 0.0285 & 97.56\% \\
- 文本特征 & 0.0287 & 97.52\% \\
- 对数损失函数(使用MSE) & 0.0298 & 96.85\% \\
\bottomrule
\end{tabular}
\end{table}

消融实验表明:
\begin{enumerate}[leftmargin=*]
    \item 各类特征均有正向贡献,其中文本特征的贡献最大;
    \item 对数尺度损失函数相比传统MSE有显著优势,验证了针对长尾分布设计损失函数的必要性。
\end{enumerate}

% ==================== 6. 结论 ====================
\section{结论}

本文针对社交媒体评论热度预测问题,以小米SU7微博数据为研究对象,提出了一种基于NGBoost的概率预测方法。主要结论如下:

\begin{enumerate}[leftmargin=*]
    \item 构建了包含27万条评论的数据集,设计了完整的数据采集、清洗和划分流程,为社交媒体分析研究提供了数据基础;
    \item 提出了四类互补的特征工程方案,其中基于MinHash的时间密度特征能够高效检测重复和相似评论;
    \item NGBoost模型在验证集上达到了97.61\%的ACP@20\%准确率和94.14\%的PICP@95\%覆盖率,有效实现了预测精度与不确定性校准的平衡;
    \item 特征重要性分析表明,用户活跃度、评论层级和微博热度是影响评论热度的关键因素。
\end{enumerate}

未来工作可以从以下方向展开:(1)引入更多的用户画像特征,如社交网络结构、历史互动模式等;(2)探索时序建模方法,捕捉评论热度的动态演化规律;(3)将方法推广到其他社交媒体平台,验证模型的泛化能力。

% ==================== 参考文献 ====================
\bibliographystyle{plain}
\begin{thebibliography}{99}

\bibitem{liu2019sentiment}
Liu, B. (2019). Sentiment analysis: Mining opinions, sentiments, and emotions. Cambridge University Press.

\bibitem{zhang2020brand}
Zhang, L., \& Zhang, W. (2020). Brand monitoring using social media analytics. Journal of Marketing Research, 57(4), 741-762.

\bibitem{wu2019npa}
Wu, C., Wu, F., Ge, S., et al. (2019). Neural news recommendation with multi-head self-attention. In EMNLP-IJCNLP (pp. 6389-6394).

\bibitem{yang2011patterns}
Yang, J., \& Leskovec, J. (2011). Patterns of temporal variation in online media. In WSDM (pp. 177-186).

\bibitem{bandari2012pulse}
Bandari, R., Asur, S., \& Huberman, B. A. (2012). The pulse of news in social media: Forecasting popularity. In ICWSM (pp. 26-33).

\bibitem{deng2020deep}
Deng, J., \& Xie, X. (2020). Deep attention-based popularity prediction for social media. In WWW (pp. 2822-2828).

\bibitem{gal2016uncertainty}
Gal, Y. (2016). Uncertainty in deep learning. PhD thesis, University of Cambridge.

\bibitem{blundell2015weight}
Blundell, C., Cornebise, J., Kavukcuoglu, K., \& Wierstra, D. (2015). Weight uncertainty in neural networks. In ICML (pp. 1613-1622).

\bibitem{gal2016dropout}
Gal, Y., \& Ghahramani, Z. (2016). Dropout as a Bayesian approximation: Representing model uncertainty in deep learning. In ICML (pp. 1050-1059).

\bibitem{duan2020ngboost}
Duan, T., Avati, A., Ding, D. Y., et al. (2020). NGBoost: Natural gradient boosting for probabilistic prediction. In ICML (pp. 2690-2700).

\bibitem{devlin2019bert}
Devlin, J., Chang, M. W., Lee, K., \& Toutanova, K. (2019). BERT: Pre-training of deep bidirectional transformers for language understanding. In NAACL-HLT (pp. 4171-4186).

\bibitem{liu2019roberta}
Liu, Y., Ott, M., Goyal, N., et al. (2019). RoBERTa: A robustly optimized BERT pretraining approach. arXiv preprint arXiv:1907.11692.

\bibitem{bge2023}
Xiao, S., Liu, Z., Zhang, P., \& Muennighoff, N. (2023). C-Pack: Packaged resources to advance general Chinese embedding. arXiv preprint arXiv:2309.07597.

\bibitem{blei2003latent}
Blei, D. M., Ng, A. Y., \& Jordan, M. I. (2003). Latent Dirichlet allocation. Journal of Machine Learning Research, 3, 993-1022.

\bibitem{broder1997resemblance}
Broder, A. Z. (1997). On the resemblance and containment of documents. In Compression and Complexity of Sequences (pp. 21-29).

\end{thebibliography}

\end{document}
